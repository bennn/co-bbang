\documentclass{article}

%%package
\usepackage{amsmath}
\usepackage{amssymb}
\usepackage{amsthm}
\usepackage{palatino}
\usepackage{tikz}

%% def
\newcommand{\atoms}{\textsf{\small{At}}}
\newcommand{\actions}{\Sigma}
\newcommand{\gs}{\textbf{GS}}
\newcommand{\mat}[1]{\textbf{Mat}(#1)}

\newcommand{\todo}[1]{\textbf{TODO:~#1}}

\begin{document}

\section*{KAT + B!}

First we define the KAT.
Let $K = (\sigma_K, B_K)$ be the free KAT over symbols $\sigma_K$ and tests $B_K$; additionally let $\atoms_K$ be the set of atoms of the boolean algebra $B_K$ and $\gs_K$ be the guarded strings of $K$.

Next we define $B!$.
Let $F_n$ be the free boolean algebra generated by the set of tests $B!$, and let $\atoms_F$ be the atoms of this algebra.

The plan is for actions of the category formed by KAT+B! to be the actions of the KAT.
The mutable tests will be observations.
Form a category as follows:
\begin{itemize}
\item [\textbf{Obj}:]
  The objects are the matrices in $\mat{B!,~K}$.
  The KAT+B! paper says that these matrices are isomorphic to the commutative coproduct representing the KAT+B!, so they seem the best representation combining both KAT terms and mutable tests into each object.
  (Another idea is tuples of $(\atoms_K,!\atoms_F)$, but this is awkward and I think insufficient.)
\item [\textbf{id}:]
  The identity arrows are the identity arrows of the KAT and the identity arrows of the mutable tests $B!$.
  \todo{each object must have exactly one identity! Compose arrows?}
\item [$\mathcal{O}$:]
  Observations are these identity arrows augmented with observations of the mutable tests.
  That is, $\textbf{id} \cup \{ (b?,c!)~|~b,c \in B! \}$.
\item [$\Sigma$:]
  Actions are the actions from the KAT, guarded strings $\alpha x \beta$.
\end{itemize}

From this definition of actions, composition, co-multiplication, and co-iteration are exactly the same as for the category of guarded strings.
This makes sense \textemdash mutable tests and checks may be interspersed within an expression without affecting the behavior of the KAT terms.
We've just enhanced the KAT with additional observations.

\end{document}
